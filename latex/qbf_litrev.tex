\documentclass{article}
\usepackage[utf8]{inputenc}
\usepackage[margin=2cm]{geometry}
%\usepackage{enumitem}
\usepackage{csquotes}

% maths packages
\usepackage{amsmath}
\usepackage{amssymb}
\usepackage{amsthm}
\usepackage{braket}

% remove space between ket and bra
\renewcommand\bra[1]{{\langle{#1}|}}
\makeatletter
\renewcommand\ket[1]{
  \@ifnextchar\bra{\k@t{#1}\!}{\k@t{#1}}
}
\renewcommand\ket[1]{
  \@ifnextchar\braket{\k@t{#1}\!}{\k@t{#1}}
}
\newcommand\k@t[1]{{|{#1}\rangle}}
\makeatother

% graphics packages
\usepackage{graphicx}
%\usepackage{subfig}
\usepackage[font={small,it}]{caption}
\usepackage{tikz}
\usetikzlibrary{positioning, circuits.logic.US}

\title{QBF lit review?}
\author{Suzie Brown}
\date{\today}

% bibliography
\usepackage[round, sort&compress]{natbib}
\usepackage{har2nat} %%% Harvard reference style
\bibliographystyle{agsm}

% theorems
\newtheorem{thm}{Theorem}
\theoremstyle{definition}
\newtheorem{defn}{Definition}
\newtheorem{example}{Example}

% probability symbols
\newcommand{\PR}{\mathbb{P}}
\newcommand{\E}{\mathbb{E}}
\newcommand{\V}{\operatorname{Var}}
\newcommand{\iidsim}{\overset{iid}{\sim}}
\newcommand{\eqdist}{\overset{d}{=}}

% distributions
\newcommand{\Bern}{\operatorname{Bernoulli}}
\newcommand{\Geom}{\operatorname{Geom}}

% project-specific commands
\newcommand{\A}{\mathcal{A}}
\newcommand{\AND}{{\footnotesize AND }}
\newcommand{\NAND}{{\footnotesize NAND }}
\newcommand{\OR}{{\footnotesize OR }}
\newcommand{\NOR}{{\footnotesize NOR }}
\newcommand{\NOT}{{\footnotesize NOT }}

\begin{document}
%\maketitle

\section{Quantum information}
From the course by Noah Linden (Bristol) and discussion with Thomas Hebdige and David Jennings (Imperial). For a comprehensive introduction see for example \citet{nielsen2002} or \citet{wilde2013}.\\

Quantum mechanics is essentially just linear algebra in a Hilbert space, with a few additional properties. The following definitions are exactly what you would find in a linear algebra course, apart from the notation.

\subsection{Dirac notation}
Dirac notation is a convenient way of denoting vectors such that it is easy to visually identify inner and outer products, and thus quickly recognise scalars, vectors and matrices:

\begin{itemize}
\item $\ket{v}$ denotes a column vector
\item $\braket{u|v}$ denotes an inner product (resulting in a scalar)
\item $\ket{u}\bra{v}$ denotes an outer product (resulting in a matrix)
\end{itemize}
Additionally, $\overline{\alpha}$ denotes the complex conjugate of a scalar $\alpha$.

\subsection{Hilbert space}
A \emph{Hilbert space} is a vector space with an inner product $\braket{\cdot|\cdot}$ satisfying the following:
\begin{itemize}
\item $\bra{u} (\alpha\ket{v} + \beta\ket{w}) = \alpha\braket{u|v} + \beta\braket{u|w}$
\item $\braket{u|v} = \overline{\braket{v|u}}$
\item $\braket{v|v} \geq 0$ with equality iff $\ket{v}$ is the zero vector.
\end{itemize}

\subsection{Orthonormal bases}
An \emph{orthonormal basis} of a Hilbert space $\mathcal{H}$ is a set of vectors $\{v_1,\dots,v_n\}$ in $\mathcal{H}$ such that:
\begin{itemize}
\item $\operatorname{span}(\{v_1,\dots,v_n\}) = \mathcal{H}$
\item $\braket{v_i|v_j} = \delta_{ij}$
\end{itemize}
Restricting to the space $\mathbb{C}^2$, which is all that is needed to understand the quantum Bernoulli factory, we have the \emph{computational basis} $\{\ket{0}, \ket{1}\}$. This is the canonical basis and is henceforth used wherever not specified otherwise. Since it is an orthonormal basis, every vector $\ket{v}$ in $\mathbb{C}^2$ has a unique representation
\begin{equation*}
\ket{v} = \alpha \ket{0} + \beta \ket{1} \equiv (\alpha,\beta)^T
\end{equation*}
for some $\alpha, \beta \in \mathbb{C}$. For reasons which will probably not become apparent in this treatment, we will restrict to \emph{normalised} vectors, requiring also $|\alpha|^2 + |\beta|^2 = 1$.
To ensure coherency with the properties of the inner product, we have that
\begin{equation*}
\bra{v} = \overline{\alpha} \bra{0} + \overline{\beta} \bra{1}.
\end{equation*}
The inner product of $\ket{u} = u_0\ket{0} + u_1\ket{1}$ with $\ket{v} = v_0\ket{0} + v_1\ket{1}$ is therefore computed as
\begin{align*}
\braket{u|v} &= (\overline{u_0}\bra{0} + \overline{u_1}\bra{1}) (v_0\ket{0} + v_1\ket{1})\\
&= \overline{u_0}v_0\braket{0|0} + \overline{u_0}v_1\braket{0|1} + \overline{u_1}v_0\braket{1|0} + \overline{u_1}v_1\braket{1|1} \\
&= \overline{u_0}v_0 + \overline{u_1}v_1.
\end{align*}
One alternative choice of orthonormal basis which is worth mentioning is given by $\{\ket{+},\ket{-}\}$, consisting of the states
\begin{align*}
&\ket{+} := \frac{1}{\sqrt{2}}(\ket{0} + \ket{1}) \\
&\ket{-} := \frac{1}{\sqrt{2}}(\ket{0} - \ket{1}).
\end{align*}

\subsection{Linear operators}
A linear operator is an operator with the property
\begin{equation*}
A(\alpha\ket{u} + \beta\ket{v}) = \alpha A\ket{u} + \beta A\ket{v}.
\end{equation*}
It is therefore fully defined according to its action on an orthonormal basis. For instance, the quantum \NOT operator (denoted $X$) is defined by
\begin{align*}
& X\ket{0} = \ket{1} \\
& X\ket{1} = \ket{0}
\end{align*}
Equivalently, $X$ can be expressed as a matrix with respect to the computational basis:
\begin{equation*}
X = \left(\begin{matrix}
0 & 1 \\ 1 & 0
\end{matrix}\right)
\end{equation*}
In Dirac notation, X can be written in terms of outer products of basis states:
\begin{equation*}
X = \ket{0}\bra{1} + \ket{1}\bra{0}
\end{equation*}
Then if X acts on the state $\ket{v} = \alpha \ket{0} + \beta \ket{1}$, we have
\begin{align*}
X\ket{v} &= (\ket{0}\bra{1} + \ket{1}\bra{0})(\alpha \ket{0} + \beta \ket{1})\\
&= \alpha \ket{0}\braket{1|0} + \alpha \ket{1}\braket{0|0} + \beta \ket{0}\braket{1|1} +
\beta \ket{1}\braket{0|1}\\
&= \alpha \ket{1} + \beta \ket{0}
\end{align*}
as desired.


\bibliography{qbf.bib}
\end{document}