\documentclass{article}
\usepackage[utf8]{inputenc}
\usepackage[margin=2cm]{geometry}
\usepackage{amsmath}
\usepackage{amssymb}
\usepackage{amsthm}
%\usepackage{graphicx}
%\usepackage{subfig}
%\usepackage{enumitem}

\title{Quantum Bernoulli Factory}
\author{Suzie Brown}
\date{\today}

%\usepackage[colorlinks=true, allcolors=blue]{hyperref}
\usepackage[round, sort&compress]{natbib}
\usepackage{har2nat} %%% Harvard reference style
\bibliographystyle{agsm}

%\newcommand{\E}{\mathbb{E}}
%\newcommand{\PR}{\mathbb{P}}
%\newcommand{\V}{\operatorname{Var}}
\newcommand{\Bern}{\operatorname{Bernoulli}}
\newcommand{\A}{\mathcal{A}}
\newcommand{\iidsim}{\overset{iid}{\sim}}

%\newtheorem{thm}{Theorem}
\theoremstyle{definition}
\newtheorem{defn}{Definition}
\newtheorem{example}{Example}

\begin{document}
\maketitle
\section*{Problem statement}
Suppose we have access to a black box producing coin flips where the probability of observing heads is $p$. Roughly speaking, a Bernoulli factory is an algorithm that uses queries of this black box to produce a coin flip where the probability of observing heads is $f(p)$, for some specified function $f$.

Let us now make this notion more precise.
\begin{defn}\label{defn:bern_fact}
Let $f: S\to[0,1]$ be a function with domain $S \subseteq [0,1]$, and suppose we have a sequence of Bernoulli random variables $X_1,X_2,\dots \iidsim \Bern(p)$ with unknown parameter $p \in [0,1]$. 
Let $U \in \mathcal{U}$ denote a set of auxiliary random variables with known distributions, independent of $p$. Let $\tau(U)$ be a stopping time with respect to the natural filtration.
A \emph{Bernoulli factory} is a function $\A : \mathcal{U} \times \{0,1\}^T \to\{0,1\}$ such that $\A(U,X_1,\dots,X_T) \sim \Bern(f(p))$ for all $p \in [0,1]$.
% for all U?
\end{defn}
% define T
% this is a special case of the `simulability' definition given in Keane1994.
% obviously must restrict to p \in (0,1) for sensible cases.

\begin{example}
Hello world
\end{example}


%\citet{dale2015}


\bibliography{qbf.bib}
\end{document}